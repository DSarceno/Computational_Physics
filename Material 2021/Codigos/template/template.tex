\documentclass[letterpaper]{article}

% COMPILAR USANDO

% pdflatex template
% bibtex template
% pdflatex template
% pdflatex template

% para que funcione todo en español
\usepackage[spanish,es-nodecimaldot]{babel}
% para que funcionen bien las tildes. OJO ver si hay que cambiar la
% codificación de UTF8 a la que use el editor de texto
\usepackage[utf8]{inputenc}
% para las imágenes y figuras
\usepackage{graphicx}

% más símbolos en LaTeX
\usepackage{latexsym}%
% para usar fuentes escalables, es útil para ver el texto más grande y sin
% pérdidas
\usepackage[T1]{fontenc}
% notación matemática
\usepackage{amsmath, amsfonts, amsmath}%
% para unidades del sistema internacional
\usepackage[amssymb]{SIunits}%

% para los metadatos del PDF
\usepackage[%
bookmarksnumbered,%
pdfauthor={Giovanni Ramirez (ramirez@ecfm.usac.edu.gt)},%
pdftitle={Template},%
pdfsubject={Template},%
pdfkeywords={template, template}]{hyperref}

\title{\sc Template}%
\author{Giovanni Ramírez \thanks{Escuela de Ciencias Físicas y Matemáticas,
  Universidad de San Carlos de Guatemala.}}
\date{Guatemala, \today}
%% 20210307

\begin{document}
\maketitle

\begin{abstract}
  bla bla bla bla bla bla bla bla bla bla bla bla bla bla bla bla bla bla bla
  bla bla bla bla bla bla bla bla bla bla bla bla bla bla bla bla bla bla bla
  bla bla bla bla bla bla bla bla bla bla bla bla bla bla bla bla bla bla bla
  bla bla bla bla bla bla bla bla bla bla bla bla

  bla bla bla bla bla bla bla bla bla bla bla bla bla bla bla bla bla bla bla
  bla bla bla bla bla bla bla bla bla bla bla bla bla bla bla bla bla bla bla
  bla bla bla bla bla bla bla bla bla bla bla bla bla bla bla bla bla bla bla
  bla bla bla bla bla bla bla bla bla bla bla bla 
\end{abstract}


\section{Introducción}
\label{sec:intro}
bla bla bla bla bla bla bla bla bla bla bla bla bla bla bla bla bla bla bla
bla bla bla bla bla bla bla bla bla bla bla bla bla bla bla bla bla bla bla
bla bla bla bla bla bla bla bla bla bla bla bla bla bla bla bla bla bla bla
bla bla bla bla bla bla bla bla bla bla bla bla
% 
\begin{displaymath}
  \int_a^b f(x) dx,
\end{displaymath}
%
bla bla bla bla bla bla bla bla bla bla bla bla bla bla bla bla bla bla bla
bla bla bla bla bla bla bla bla bla bla bla bla bla bla bla bla bla bla bla
bla bla bla bla bla bla bla bla bla bla bla bla bla bla bla bla bla bla bla
bla bla bla bla bla bla bla bla bla bla bla bla.


\section{Método}
\label{sec:metodo}
bla bla bla bla bla bla bla bla bla bla bla bla bla bla bla bla bla bla bla
bla bla bla bla bla bla bla bla bla bla bla bla bla bla bla bla bla bla bla
bla bla bla bla bla bla bla bla bla bla bla bla bla bla bla bla bla bla bla
bla bla bla bla bla bla bla bla bla bla bla bla

bla bla bla bla bla bla bla bla bla bla bla bla bla bla bla bla bla bla bla
bla bla bla bla bla bla bla bla bla bla bla bla bla bla bla bla bla bla bla
bla bla bla bla bla bla bla bla bla bla bla bla bla bla bla bla bla bla bla
bla bla bla bla bla bla bla bla bla bla bla bla

bla bla bla bla bla bla bla bla bla bla bla bla bla bla bla bla bla bla bla
bla bla bla bla bla bla bla bla bla bla bla bla bla bla bla bla bla bla bla
bla bla bla bla bla bla bla bla bla bla bla bla bla bla bla bla bla bla bla
bla bla bla bla bla bla bla bla bla bla bla bla

bla bla bla bla bla bla bla bla bla bla bla bla bla bla bla bla bla bla bla
bla bla bla bla bla bla bla bla bla bla bla bla bla bla bla bla bla bla bla
bla bla bla bla bla bla bla bla bla bla bla bla bla bla bla bla bla bla bla
bla bla bla bla bla bla bla bla bla bla bla bla

bla bla bla bla bla bla bla bla bla bla bla bla bla bla bla bla bla bla bla
bla bla bla bla bla bla bla bla bla bla bla bla bla bla bla bla bla bla bla
bla bla bla bla bla bla bla bla bla bla bla bla bla bla bla bla bla bla bla
bla bla bla bla bla bla bla bla bla bla bla bla

bla bla bla bla bla bla bla bla bla bla bla bla bla bla bla bla bla bla bla
bla bla bla bla bla bla bla bla bla bla bla bla bla bla bla bla bla bla bla
bla bla bla bla bla bla bla bla bla bla bla bla bla bla bla bla bla bla bla
bla bla bla bla bla bla bla bla bla bla bla bla
%
\begin{displaymath}
  \int_a^b f(x) dx \approx (b-a) \left( \frac{f(a)+f(b)}{2} \right).
\end{displaymath}

\begin{figure}
  \centering
  %\includegraphics[width=.45\columnwidth]{images/golubTrapezoidal}
  %\includegraphics[width=.45\columnwidth]{images/cohenTrapezoidal}
  \caption{Método del trapecio para aproximar el área bajo la curva. A la
    izquierda mediante el área de un trapecio. A la derecha se muestra el
    detalle del $(k+1)$-ésimo trapecio cuando se usan $N$ trapecios en la
    aproximación, aquí se muestra con énfasis el error de aproximación.
    Fuente: la figura de la izquierda aparece en Golub y Ortega \cite{golub},
    cap 5.  La figura de la derecha aparece en Cohen \cite{cohen}, cap. 4.}
  \label{fig:curva}
\end{figure}


bla bla bla bla bla bla bla bla bla bla bla bla bla bla bla bla bla bla bla
bla bla bla bla bla bla bla bla bla bla bla bla bla bla bla bla bla bla bla
bla bla bla bla bla bla bla bla bla bla bla bla bla bla bla bla bla bla bla
bla bla bla bla bla bla bla bla bla bla bla bla

bla bla bla bla bla bla bla bla bla bla bla bla bla bla bla bla bla bla bla
bla bla bla bla bla bla bla bla bla bla bla bla bla bla bla bla bla bla bla
bla bla bla bla bla bla bla bla bla bla bla bla bla bla bla bla bla bla bla
bla bla bla bla bla bla bla bla bla bla bla bla


\section{Resultados}
\label{sec:resultados}
bla bla bla bla bla bla bla bla bla bla bla bla bla bla bla bla bla bla bla
bla bla bla bla bla bla bla bla bla bla bla bla bla bla bla bla bla bla bla
bla bla bla bla bla bla bla bla bla bla bla bla bla bla bla bla bla bla bla
bla bla bla bla bla bla bla bla bla bla bla bla

bla bla bla bla bla bla bla bla bla bla bla bla bla bla bla bla bla bla bla
bla bla bla bla bla bla bla bla bla bla bla bla bla bla bla bla bla bla bla
bla bla bla bla bla bla bla bla bla bla bla bla bla bla bla bla bla bla bla
bla bla bla bla bla bla bla bla bla bla bla bla

\begin{figure}
  \centering
  %\includegraphics[width=.85\textwidth]{images/TrapSerr}  
  \caption{Error relativo de la aproximación en función del número de
    trapecios. Fuente: elaboración propia.}
  \label{fig:Serror}
\end{figure}


bla bla bla bla bla bla bla bla bla bla bla bla bla bla bla bla bla bla bla
bla bla bla bla bla bla bla bla bla bla bla bla bla bla bla bla bla bla bla
bla bla bla bla bla bla bla bla bla bla bla bla bla bla bla bla bla bla bla
bla bla bla bla bla bla bla bla bla bla bla bla

bla bla bla bla bla bla bla bla bla bla bla bla bla bla bla bla bla bla bla
bla bla bla bla bla bla bla bla bla bla bla bla bla bla bla bla bla bla bla
bla bla bla bla bla bla bla bla bla bla bla bla bla bla bla bla bla bla bla
bla bla bla bla bla bla bla bla bla bla bla bla



\section{Conclusiones}
\label{sec:conclusiones}

bla bla bla bla bla bla bla bla bla bla bla bla bla bla bla bla bla bla bla
bla bla bla bla bla bla bla bla bla bla bla bla bla bla bla bla bla bla bla
bla bla bla bla bla bla bla bla bla bla bla bla bla bla bla bla bla bla bla
bla bla bla bla bla bla bla bla bla bla bla bla

bla bla bla bla bla bla bla bla bla bla bla bla bla bla bla bla bla bla bla
bla bla bla bla bla bla bla bla bla bla bla bla bla bla bla bla bla bla bla
bla bla bla bla bla bla bla bla bla bla bla bla bla bla bla bla bla bla bla
bla bla bla bla bla bla bla bla bla bla bla bla

\begin{figure}
  \centering
  % \includegraphics[width=.85\textwidth]{images/TrapMerr}  
  \caption{Área bajo la curva obtenido medidante las dos implementaciones,
    secuencial y paralelo en función del número de trapecios.  En el inserto
    está la diferencia del valor obtenido en la implementación secuencial
    menos el valor obtenido en la implementación paralela.  Fuente:
    elaboración propia.}
  \label{fig:Merror}
\end{figure}


bla bla bla bla bla bla bla bla bla bla bla bla bla bla bla bla bla bla bla
bla bla bla bla bla bla bla bla bla bla bla bla bla bla bla bla bla bla bla
bla bla bla bla bla bla bla bla bla bla bla bla bla bla bla bla bla bla bla
bla bla bla bla bla bla bla bla bla bla bla bla

bla bla bla bla bla bla bla bla bla bla bla bla bla bla bla bla bla bla bla
bla bla bla bla bla bla bla bla bla bla bla bla bla bla bla bla bla bla bla
bla bla bla bla bla bla bla bla bla bla bla bla bla bla bla bla bla bla bla
bla bla bla bla bla bla bla bla bla bla bla bla

\section*{Agradecimientos}
\label{sec:agradecimientos}

Se agradece a la ECFM-USAC por el uso del clúster Euclides donde se realizaron
las pruebas de rendimiento reportadas en este trabajo.

% References
\nocite{*}
\bibliographystyle{unsrt}%
\bibliography{references}%
\end{document}


%%% Local Variables:
%%% mode: latex
%%% TeX-master: t
%%% End:
