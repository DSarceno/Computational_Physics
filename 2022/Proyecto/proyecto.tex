\input{Preamble_general.tex}




% para los metadatos del PDF
%\usepackage[%
%bookmarksnumbered,%
%pdfauthor={Diego Sarceño (dsarceno68@gmail.com)},%
%pdftitle={Puntos de Lagrange},%
%pdfsubject={Proyecto},%
%pdfkeywords={template, template}]{hyperref}

\title{\sc Problema de los $N$ Cuerpos \\ \footnotesize{Física Computacional}}%
\author{Diego Sarceño \\ $201900109$}
\date{Guatemala, \today}
%% 20210307

\begin{document}  
\maketitle

\begin{abstract}
	\lipsum[1]
\end{abstract}


%\section{Introducción}
%\label{sec:intro}
%\justify 
%\lipsum[1]	


\section{Problema de los $N$ Cuerpos}
\label{sec:Problema de los $N$ Cuerpos}
\justify
Las primeras ideas acerca del "problema de los $n$ cuerpos" llegaron a comienzos del siglo XVII de la mano de Newton al decir que no es suficiente con especificar la posición inicial y la velocidad, o tampoco tres posiciones orbitales, sino que, para determinar con certeza la orbita de un planeta es necesario tomar en cuenta las fuerzas gravitatorias iteractivas. Newton no lo plantea directamente, pero en sus "Principia" se deduce que el problema de los $n$ es irresoluble.

Ya se ha visto en otros cursos (Mecánica Clásica 1 y 2) que simplificaciones de este problema son posibles de resolver o aproximar analíticamente. Estas son: El problema de los $2$ cuerpos y el problema de los $3$ cuerpos restringido. En este caso se tomarán todas las masas involucradas iguales ($10^{18} kg$), además de restringir el movimiento de las mismas al plano $xy$.









\section{Implementación}
\label{sec:implementacion}

\subsection{Posiciones Iniciales}
Para las posiciones iniciales se utilizó una distribución uniforme de números en el rango de $[-1,1]$, en unidades astronómicas ($1UA = 1.5\times 10^{11} m$). Para esto se utilizaron librerías como \texttt{<cstdlib>} y \texttt{<random>}; así como una semilla (seed) para que las pruebas realizadas se puedan replicar la simulación. Con esto se generan los arreglos para las coordenadas $x$ e $y$.


\subsection{Velocidades Iniciales}
Dado que la distribución usada para colocar las partículas en la región dada es uniforme, entre más partículas se coloquen en dicha distribución, mejor será la aproximación a una distribución homogenea de masa. Teniendo esto, se supone una trayectoria circular inical, entonces igualando la fuerza centrípeta y gravitacional, se tiene
\begin{equation}
	\frac{m_i v_i ^2}{r_i} = \frac{GMm_i}{r_i ^2}, \qquad \qquad r_i = \sqrt{x_{i0} ^2 + iy_{i0} ^2},  \label{circular}
\end{equation}
con $M$ es la masa contenida en dicho circulo inicial, esta se puede calcular utilizando la densidad superficial de masa $\flatfrac{Nm_i}{L^2}$. Con esto se obtiene que
\begin{equation}
	v_i = \frac{\sqrt{G\pi Nm_i r_i}}{L}. \nonumber
\end{equation}
Además, es necesario añadirle la dirección aleatoria la cuál está dada por un vector unitario generado por la posición de la partícula
\begin{equation}
	\vb{v}_{i0} = \frac{\sqrt{G\pi Nm_i r_i}}{L} \qty(-\frac{y_{i0}}{r_i},\frac{x_{i0}}{r_i}). \label{velocidades}
\end{equation}
Aparte de esto, a cada partícula se le añade un valor adicional en cada dirección.



\subsection{Ecuaciones de Movimiento}
Para cada partícula se tiene una fuerza neta actuando sobre ella, con la ventaja de ser un sistema aislado. Cada una de estas fuerzas incluye la interacción del resto de partículas, por ende
\begin{equation}
	\vec{F}_i = \sum _{i\neq j} -Gm_i m_j \frac{\vec{r}_i - \vec{r} _j}{\abs{\vec{r}_i - \vec{r} _j}^3} \label{fuerza}
\end{equation}
con $\abs{\vec{r}_i - \vec{r} _j}^2 = (x_i - x_j)^2 + (y_i - y_j)^2$. Entonces, por segunda ley de Newton se tienen las siguientes ecuaciones de movimiento
\begin{align*}
	\ddot{x} _i &= \sum _{i\neq j} -\frac{Gm_j}{\abs{\vec{r}_i - \vec{r} _j}} (x_i - x_j), \\
	 \ddot{y} _i &= \sum _{i\neq j} -\frac{Gm_j}{\abs{\vec{r}_i - \vec{r} _j}} (y_i - y_j).
\end{align*}

\subsection{Energía y Momentum}
Utilizando las definiciones clásicas para muchas partículas
	\begin{align*}
		E &= \frac{1}{2} \sum _{i} ^N m_i v_i ^2 - G\sum_{i < j} \frac{m_i m_j}{r_{ij}} \\
		P &= \sum _i ^N m_i v_i .
	\end{align*}

\subsection{Escritura de los Archivos}
La parte más importante antes de las gráficas, los datos devueltos por el programa. A lo largo del desarrollo del proyecto, esta fue una de las partes en las que se presentaron más problemas, recibí ayuda para poder tener un buen output a mi programa. Se utilizó la clase \textit{StringStream} para ello.


\section{Resultados}
\label{sec:resultados}
El código complió bien; sin embargo, no dio resultados satisfactorios. Las hojas de datos mostradas por el programa daban resultados inmanejables y/o sin sentido físico. Obviamente es un error en alguna de las funciones, pero por mala planificación del tiempo por mi parte no pude encontrar dicho error.


%\section{Discución de Resultados}
%\label{sec:discucion}
%
%\lipsum[1]
%
%
%\section{Conclusiones}
%\label{sec:conclusiones}
%
%\lipsum[1]


\section{Anexos}
\label{sec:anexos}
\subsection{Código}
\subsubsection{Prueba para $3$ Cuerpos}
\begin{lstlisting}
// Librerias
#include <iostream>
#include <cmath>
#include <iomanip>
#include <fstream>
#include <cstdlib>
#include <random>


using namespace std;


std::default_random_engine generator(17); // seed
std::uniform_real_distribution<long double> random_interval(-1.0,1.0);







/*
// constantes globales
#define n_cuerpos (100)
#define n_ec (n_cuerpos*4)
#define G (6.67e-11)
*/

void RK4( double *y,
	  const int n_ec,
	  const double t,
	  const double h,
	  double *y_imas1,
	  void (*derivada)( double *, const double, double * ) );


const int n_cuerpos = 3;
const int n_ec = 4*n_cuerpos;
const double G = 6.67e-11;
// Variables globales
double *xp, *yp; // posicion en x, y
double *vx, *vy; // velocidad en x, y
double ax[n_cuerpos], ay[n_cuerpos]; // aceleracion en x, y
double masa[n_cuerpos]; // masa de cada particula
double E, P, L;






// Inicializar masas
void init_masa(){
	masa[0] = 1.989e30;
  masa[1] = 5.972e24;
  masa[2] = 7.348e22;
} // END masa

// Inicializar posiciones
void init_posicion(){
  xp[0] = 0.0;
  xp[1] = 1.5096e11;
  xp[2] = 1.5134e11;
  yp[0] = 0.0;
  yp[1] = 0.0;
  yp[2] = 0.0;
} // END init_posicion

  // Inicializar velocidades
void init_velocidad(){
	vx[0] = 0.0;
  vx[1] = 0.0;
  vx[2] = 0.0;
  vy[0] = 0.0;
  vy[1] = 2.979e4;
  vy[2] = 3.080e4;
} // END init_velocidad


void calc_aceleracion(){
	for (int i = 0; i < n_cuerpos; i++){
		ax[i] = 0;
		ay[i] = 0;
		for (int j = 0; j < n_cuerpos; j++){
			double deltaX = xp[i] - xp[j];
			double deltaY = yp[i] - yp[j];
			double common_term = G*masa[j]*pow(pow(deltaX, 2) + pow(deltaY, 2), -3/2);
			ax[j] -= common_term*deltaX;
			ay[i] -= common_term*deltaY;
		} // END FOR
	} // END FOR
} // END calc_aceleracion




// Derivadas del sistemas de ecuaciones
void nCuerposGrav(double y[n_ec], const double t, double dydt[n_ec]){
  xp   = y;
  yp   = y +   n_cuerpos;
  vx   = y + 2*n_cuerpos;
  vy   = y + 3*n_cuerpos;

  // Calcular aceleraciones
  calc_aceleracion();

  for( int i=0; i<n_cuerpos; i++ ){
    dydt[i              ] = vx[i];
    dydt[i +   n_cuerpos] = vy[i];
    dydt[i + 2*n_cuerpos] = ax[i];
    dydt[i + 3*n_cuerpos] = ay[i];
  } // END FOR
} // END nCuerposGrav


void energia_momentos(){
	E = 0.0;
	L = 0.0;
	P = 0.0;
	double energia_potencial = 0.0;
	double energia_cinetica = 0.0;
	for (int i = 0; i < n_cuerpos; i++){
		P += masa[i]*sqrt(vx[i]*vx[i] + vy[i]*vy[i]);
		energia_cinetica += 0.5*masa[i]*(vx[i]*vx[i] + vy[i]*vy[i]);
		for (int j = 0; j < i; j++){
			energia_potencial += G*masa[i]*masa[j]/sqrt(pow(xp[i] - xp[j],2) + pow(yp[i] - yp[j],2));
		} // END FOR
	} // END FOR
	E = energia_cinetica - energia_potencial;
} // END energia_momentos



void archivo(ofstream &of, stringstream &ss){
	of << ss.str();
}

void salidaSolucion(const double &t, stringstream &ss){
	ss << t;
	for (int i = 0; i < n_cuerpos; i++){
		ss << "\t" << xp[i] << "\t" << yp[i];
	} // END FOR
	ss << endl;
} // END salidaSolucion


void salidaVelocidad(const double &t, stringstream &ss){
	ss << t;
	for (int i = 0; i < n_cuerpos; i++){
		ss << "\t" << xp[i] << "\t" << yp[i];
	} // END FOR
	ss << endl;
} // END salidaVelocidad





void salidaEnergiaMomento(const double &t, stringstream &ss){
	ss << t << "\t" << E << "\t" << P << endl;
} // END salidaEnergia



/*
void salidaMomentumLineal(const double tiempo, ofstream &of){

} // END salidaMomentumLineal




void salidaMomentumAngular(const double tiempo, ofstream &of){

} // END salidaMomentumAngular
*/



void salidaMasa(const double &t, stringstream &ss){
	ss << t;
	for (int i = 0; i < n_cuerpos; i++){
		ss << "\t" << masa[i];
	} // END FOR
	ss << endl;
} // END salidaMasa



void verificar_colisiones( double t )
{
  double dist_lim = 1.0e7;

  for( int i=0; i<n_cuerpos; i++ ){
    if ( masa[i] != 0.0 ){
      for( int j=0; j<i; j++ ){
	if ( masa[j] != 0.0 ){
	  double dx = xp[i]-xp[j];
	  double dy = yp[i]-yp[j];
	  double distancia = sqrt( dx*dx + dy*dy );
	  if ( distancia < dist_lim ){
	    vx[i] = (masa[i]*vx[i] + masa[j]*vx[j] ) / (masa[i]+masa[j]);
	    vy[i] = (masa[i]*vy[i] + masa[j]*vy[j] ) / (masa[i]+masa[j]);
	    masa[i] = masa[i] + masa[j];

	    // particula j sigue la misma trayectoria que particula i pero sin masa
	    vx[j] = 0.0;
	    vy[j] = 0.0;
	    masa[j] = 0.0;
	    cout << "Colision "<< i <<" "<< j << " en t = " << t << endl;
	  }
	}
      }
    }
  }
}



void RK4( double *y,
	  const int n_ec,
	  const double t,
	  const double h,
	  double *y_imas1,
	  void (*derivada)( double *, const double, double * ) )
{
  double *k0 = new double[ n_ec ];
  double *k1 = new double[ n_ec ];
  double *k2 = new double[ n_ec ];
  double *k3 = new double[ n_ec ];
  double *z  = new double[ n_ec ];

  (*derivada)( y, t, k0 );

  for( int i=0; i<n_ec; i++ )
    z[i] = y[i] + 0.5*k0[i]*h;


  (*derivada)( z, t+0.5*h, k1 );

  for( int i=0; i<n_ec; i++ )
    z[i] = y[i] + 0.5*k1[i]*h;

  (*derivada)( z, t+0.5*h, k2 );

  for( int i=0; i<n_ec; i++ )
    z[i] = y[i] + k2[i]*h;

  (*derivada)( z, t+h, k3 );

  for( int i=0; i<n_ec; i++ )
   y_imas1[i] = y[i] + h/6.0 * ( k0[i] + 2*k1[i] + 2*k2[i] + k3[i] );

  delete[] k0;
  delete[] k1;
  delete[] k2;
  delete[] k3;
  delete[] z;
}




int main()
{
  // Parametros
	// de 5 dias en 5 dias
  const int Niter = 5000; // numero de iteraciones
  const double h_step = 1; // tamano de paso
  const int out_cada = 10; // output cada out_cada iteraciones

  double tiempo = 0.0;

	/*
  // Archivos de salida
  ofstream of_posicion( "posicion.dat", ios::out );
  ofstream of_velocidad( "velocidad.dat", ios::out );
  ofstream of_energia( "energia.dat", ios::out );
  ofstream of_momentumLineal( "momLineal.dat", ios::out );
  ofstream of_momentumAngular( "momAngular.dat", ios::out );
  ofstream of_masa( "masa.dat", ios::out );
	*/

  // reservar espacio para y
  double y[n_ec];
  double y_nueva[n_ec];

  // reservar espacio para posicion y velocidad
  xp   = y;
  yp   = y +   n_cuerpos;
  vx   = y + 2*n_cuerpos;
  vy   = y + 3*n_cuerpos;


  // puntero a la funcion "derivada"
  void (*derivada)( double *, const double, double * );
  derivada = nCuerposGrav;


  // inicializar y_nueva
  for( int i=0; i<n_ec; i++ ) y_nueva[i] = 0.0;



  // Inicializar masa
  init_masa();

  // Inicializar posicion
  init_posicion();

  // Inicializar velocidad
  init_velocidad();


	// strings
	std::stringstream ss_posicion;
	std::stringstream ss_velocidad;
  std::stringstream ss_energia;
  std::stringstream ss_masa;


  // ciclo de iteraciones
  for( int k=0; k<=Niter; k++ ){
    // Nombres faciles para las variables
    xp   = y;
    yp   = y +   n_cuerpos;
    vx   = y + 2*n_cuerpos;
    vy   = y + 3*n_cuerpos;


		energia_momentos();
    // Verificar colisiones
    verificar_colisiones(tiempo);



    // salida
    if (k%out_cada == 0){
      salidaSolucion(tiempo, ss_posicion);
      salidaVelocidad(tiempo, ss_velocidad);
      salidaEnergiaMomento(tiempo, ss_energia);
      salidaMasa(tiempo, ss_masa);
    }
    RK4( y, n_ec, tiempo, h_step, y_nueva, derivada );

    // Intercambiar valores
    for( int i=0; i<n_ec; i++ ) y[i] = y_nueva[i];



    // Incrementar el tiempo
    tiempo += h_step;
  }
	ofstream of_posicion("posicion.dat", ios::out);
  ofstream of_velocidad("velocidad.dat", ios::out);
  ofstream of_energia("energia_momentos.dat", ios::out);
  ofstream of_masa("masa.dat", ios::out);

  archivo(of_posicion, ss_posicion);
  archivo(of_velocidad, ss_velocidad);
  archivo(of_energia, ss_energia);
  archivo(of_masa, ss_masa);
  return 0;
}

\end{lstlisting}

\subsubsection{Problema de los $N$ Cuerpos}
\begin{lstlisting}
// Librerias
#include <iostream>
#include <cmath>
#include <iomanip>
#include <fstream>
#include <cstdlib>
#include <random>


using namespace std;


std::default_random_engine generator(17); // seed
std::uniform_real_distribution<long double> random_interval(-1.0,1.0);







/*
// constantes globales
#define n_cuerpos (100)
#define n_ec (n_cuerpos*4)
#define G (6.67e-11)
*/

void RK4( double *y,
	  const int n_ec,
	  const double t,
	  const double h,
	  double *y_imas1,
	  void (*derivada)( double *, const double, double * ) );


const int n_cuerpos = 100;
const int n_ec = 4*n_cuerpos;
const double G = 6.67e-11;
// Variables globales
double *xp, *yp; // posicion en x, y
double *vx, *vy; // velocidad en x, y
double ax[n_cuerpos], ay[n_cuerpos]; // aceleracion en x, y
double masa[n_cuerpos]; // masa de cada particula
double E, P, L;






// Inicializar masas
void init_masa(){
	for (int i = 0; i < n_cuerpos; i++){
		masa[i] = 10e18;
	} // END FOR
} // END masa

// Inicializar posiciones
void init_posicion(){
  for (int i = 0; i < n_cuerpos; i++){
		xp[i] = random_interval(generator)*1.5e11;
		yp[i] = random_interval(generator)*1.5e11;
	} // END FOR
} // END init_posicion

  // Inicializar velocidades
void init_velocidad(){
	for (int i = 0; i < n_cuerpos; i++){
		double ri = sqrt(pow(xp[i],2) + pow(yp[i],2));
		double vi = sqrt(G*M_PI*n_cuerpos*masa[i]*ri)/(2*1.5e11);

		vx[i] = vi*(-1.0*(yp[i]/ri) + random_interval(generator));
		vx[i] = vi*((xp[i]/ri) + random_interval(generator));
	} // END FOR
} // END init_velocidad


void calc_aceleracion(){
	for (int i = 0; i < n_cuerpos; i++){
		ax[i] = 0;
		ay[i] = 0;
		for (int j = 0; j < n_cuerpos; j++){
			double deltaX = xp[i] - xp[j];
			double deltaY = yp[i] - yp[j];
			double common_term = G*masa[j]*pow(pow(deltaX, 2) + pow(deltaY, 2), -3/2);
			ax[j] -= common_term*deltaX;
			ay[i] -= common_term*deltaY;
		} // END FOR
	} // END FOR
} // END calc_aceleracion




// Derivadas del sistemas de ecuaciones
void nCuerposGrav(double y[n_ec], const double t, double dydt[n_ec]){
  xp   = y;
  yp   = y +   n_cuerpos;
  vx   = y + 2*n_cuerpos;
  vy   = y + 3*n_cuerpos;

  // Calcular aceleraciones
  calc_aceleracion();

  for( int i=0; i<n_cuerpos; i++ ){
    dydt[i              ] = vx[i];
    dydt[i +   n_cuerpos] = vy[i];
    dydt[i + 2*n_cuerpos] = ax[i];
    dydt[i + 3*n_cuerpos] = ay[i];
  } // END FOR
} // END nCuerposGrav


void energia_momentos(){
	E = 0.0;
	L = 0.0;
	P = 0.0;
	double energia_potencial = 0.0;
	double energia_cinetica = 0.0;
	for (int i = 0; i < n_cuerpos; i++){
		P += masa[i]*sqrt(vx[i]*vx[i] + vy[i]*vy[i]);
		energia_cinetica += 0.5*masa[i]*(vx[i]*vx[i] + vy[i]*vy[i]);
		for (int j = 0; j < i; j++){
			energia_potencial += G*masa[i]*masa[j]/sqrt(pow(xp[i] - xp[j],2) + pow(yp[i] - yp[j],2));
		} // END FOR
	} // END FOR
	E = energia_cinetica - energia_potencial;
} // END energia_momentos



void archivo(ofstream &of, stringstream &ss){
	of << ss.str();
}

void salidaSolucion(const double &t, stringstream &ss){
	ss << t;
	for (int i = 0; i < n_cuerpos; i++){
		ss << "\t" << xp[i] << "\t" << yp[i];
	} // END FOR
	ss << endl;
} // END salidaSolucion


void salidaVelocidad(const double &t, stringstream &ss){
	ss << t;
	for (int i = 0; i < n_cuerpos; i++){
		ss << "\t" << xp[i] << "\t" << yp[i];
	} // END FOR
	ss << endl;
} // END salidaVelocidad





void salidaEnergiaMomento(const double &t, stringstream &ss){
	ss << t << "\t" << E << "\t" << P << endl;
} // END salidaEnergia



/*
void salidaMomentumLineal(const double tiempo, ofstream &of){

} // END salidaMomentumLineal




void salidaMomentumAngular(const double tiempo, ofstream &of){

} // END salidaMomentumAngular
*/



void salidaMasa(const double &t, stringstream &ss){
	ss << t;
	for (int i = 0; i < n_cuerpos; i++){
		ss << "\t" << masa[i];
	} // END FOR
	ss << endl;
} // END salidaMasa



void verificar_colisiones( double t )
{
  double dist_lim = 1.0e7;

  for( int i=0; i<n_cuerpos; i++ ){
    if ( masa[i] != 0.0 ){
      for( int j=0; j<i; j++ ){
	if ( masa[j] != 0.0 ){
	  double dx = xp[i]-xp[j];
	  double dy = yp[i]-yp[j];
	  double distancia = sqrt( dx*dx + dy*dy );
	  if ( distancia < dist_lim ){
	    vx[i] = (masa[i]*vx[i] + masa[j]*vx[j] ) / (masa[i]+masa[j]);
	    vy[i] = (masa[i]*vy[i] + masa[j]*vy[j] ) / (masa[i]+masa[j]);
	    masa[i] = masa[i] + masa[j];

	    // particula j sigue la misma trayectoria que particula i pero sin masa
	    vx[j] = 0.0;
	    vy[j] = 0.0;
	    masa[j] = 0.0;
	    cout << "Colision "<< i <<" "<< j << " en t = " << t << endl;
	  }
	}
      }
    }
  }
}



void RK4( double *y,
	  const int n_ec,
	  const double t,
	  const double h,
	  double *y_imas1,
	  void (*derivada)( double *, const double, double * ) )
{
  double *k0 = new double[ n_ec ];
  double *k1 = new double[ n_ec ];
  double *k2 = new double[ n_ec ];
  double *k3 = new double[ n_ec ];
  double *z  = new double[ n_ec ];

  (*derivada)( y, t, k0 );

  for( int i=0; i<n_ec; i++ )
    z[i] = y[i] + 0.5*k0[i]*h;


  (*derivada)( z, t+0.5*h, k1 );

  for( int i=0; i<n_ec; i++ )
    z[i] = y[i] + 0.5*k1[i]*h;

  (*derivada)( z, t+0.5*h, k2 );

  for( int i=0; i<n_ec; i++ )
    z[i] = y[i] + k2[i]*h;

  (*derivada)( z, t+h, k3 );

  for( int i=0; i<n_ec; i++ )
   y_imas1[i] = y[i] + h/6.0 * ( k0[i] + 2*k1[i] + 2*k2[i] + k3[i] );

  delete[] k0;
  delete[] k1;
  delete[] k2;
  delete[] k3;
  delete[] z;
}




int main()
{
  // Parametros
	// de 5 dias en 5 dias
  const int Niter = 73*5000; // numero de iteraciones
  const double h_step = 432000; // tamano de paso
  const int out_cada = 10000; // output cada out_cada iteraciones

  double tiempo = 0.0;

	/*
  // Archivos de salida
  ofstream of_posicion( "posicion.dat", ios::out );
  ofstream of_velocidad( "velocidad.dat", ios::out );
  ofstream of_energia( "energia.dat", ios::out );
  ofstream of_momentumLineal( "momLineal.dat", ios::out );
  ofstream of_momentumAngular( "momAngular.dat", ios::out );
  ofstream of_masa( "masa.dat", ios::out );
	*/

  // reservar espacio para y
  double y[n_ec];
  double y_nueva[n_ec];

  // reservar espacio para posicion y velocidad
  xp   = y;
  yp   = y +   n_cuerpos;
  vx   = y + 2*n_cuerpos;
  vy   = y + 3*n_cuerpos;


  // puntero a la funcion "derivada"
  void (*derivada)( double *, const double, double * );
  derivada = nCuerposGrav;


  // inicializar y_nueva
  for( int i=0; i<n_ec; i++ ) y_nueva[i] = 0.0;



  // Inicializar masa
  init_masa();

  // Inicializar posicion
  init_posicion();

  // Inicializar velocidad
  init_velocidad();


	// strings
	std::stringstream ss_posicion;
	std::stringstream ss_velocidad;
  std::stringstream ss_energia;
  std::stringstream ss_masa;


  // ciclo de iteraciones
  for( int k=0; k<=Niter; k++ ){
    // Nombres faciles para las variables
    xp   = y;
    yp   = y +   n_cuerpos;
    vx   = y + 2*n_cuerpos;
    vy   = y + 3*n_cuerpos;


		energia_momentos();
    // Verificar colisiones
    verificar_colisiones(tiempo);


		RK4( y, n_ec, tiempo, h_step, y_nueva, derivada );
    // salida
    if (k%out_cada == 0){
      salidaSolucion(tiempo, ss_posicion);
      salidaVelocidad(tiempo, ss_velocidad);
      salidaEnergiaMomento(tiempo, ss_energia);
      salidaMasa(tiempo, ss_masa);
    }


    // Intercambiar valores
    for( int i=0; i<n_ec; i++ ) y[i] = y_nueva[i];



    // Incrementar el tiempo
    tiempo += h_step;
  }
	ofstream of_posicion("posicion.dat", ios::out);
  ofstream of_velocidad("velocidad.dat", ios::out);
  ofstream of_energia("energia_momentos.dat", ios::out);
  ofstream of_masa("masa.dat", ios::out);

  archivo(of_posicion, ss_posicion);
  archivo(of_velocidad, ss_velocidad);
  archivo(of_energia, ss_energia);
  archivo(of_masa, ss_masa);
  return 0;
}

\end{lstlisting}


%\section*{Agradecimientos}
%\label{sec:agradecimientos}


% References
\nocite{*}
\bibliographystyle{IEEEannot}%
\bibliography{references}%

\begin{thebibliography}{00}
\bibitem{b1} DeVries, P. L., \& Wolf, R. P. (1994). \textit{A first course in computational physics. Computers in Physics}, 8(2), 178-179.
\bibitem{b2} Symon, K. R. (1971). \textit{Mechanics}. Addison.
\end{thebibliography}

\end{document}




%%% Local Variables:
%%% mode: latex
%%% TeX-master: t
%%% End:
