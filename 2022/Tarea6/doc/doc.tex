\input{/home/diego/Documents/Licenciatura/LatexBasic/Preamble_general}

%%%%%%%%%%%%%%%%%%%%%%%%%%%%%%%%%%%%%%%%%%%%%%%%%%%%%%%%%%%
\usepackage{fancyhdr}%formato de pagina
\pagestyle{fancy}%colocar la pagina con el formato deseado
\fancyhead{}
\fancyhead[L]{\footnotesize{Física Computacional}}
\fancyhead[C]{Tarea 6}
\fancyhead[R]{\footnotesize{\thepage}}
%\fancyhead[LO,RE]{Cálculo 3}
%\fancyhead[RO,LE]{\footnotesize{\thepage}}
\fancyfoot{}
\fancyfoot[L]{Diego Sarceño}
%\fancyfoot[LO,RE]{Diego Sarceño}
%%%%%%%%%%%%%%%%%%%%%%%%%%%%%%%%%%%%%%%%%%%%%%%%%%%%%%%%%%%
%% NUEVA BARRA INFERIOR, NICEEEE :3
\usepackage{fourier-orns}

\renewcommand\footrule{%
\hrulefill
\raisebox{-2.1pt}
{\quad\decosix\quad}%
\hrulefill}
%%%%%%%%%%%%%%%%%%%%%%%%%%%%%%%%%%%%%%%%%%%%%%%%%%%%%%%%%%%
\definecolor{DS_Black}{HTML}{000000}

\begin{document}
\begin{titlepage}
\input{Header_original}
\vspace{0.5cm}
Los códigos tanto de \textit{c++} como de \textit{gnuplot}, se pueden encontrar en la carpeta de \href{https://github.com/DSarceno/Computational_Physics/tree/main/2022/Tarea6}{Github}.

\section*{Problema 1}


\begin{lstlisting}
}
\end{lstlisting}




















%%%%%%%

\end{titlepage}
\end{document}
